% !Mode:: "Tex:UTF-8"

\section*{Advertencia inicial.}  El documento que estás utilizando es un {\bf borrador} de lo que será la versión definitiva del libro. Si este borrador tiene varios meses, es muy posible que haya una versión más reciente, disponible a través de internet (ver más abajo).

\section*{Presentación.}

Este libro nace de las clases que los autores vienen impartiendo, desde hace algunos años, en cursos de tipo {\em ``Introducción a la Estadística''}, dirigidos a estudiantes de los Grados en Biología y Biología Sanitaria de la Universidad de Alcalá. En nuestras clases hemos tratado de contar la Estadística dotándola de ``relato'', de un hilo argumental. En particular, hemos tratado de evitar una de las cosas que menos nos gustan, en muchos libros (y clases) de Matemáticas: no queremos contar la solución, antes de que el lector sepa cuál es el problema. Nos gustaría pensar que, al menos, nos hemos acercado a ese objetivo, pero serán los lectores quienes puedan juzgarlo. Al fin y al cabo, nos vamos a embarcar, con el lector, en un viaje hacia la Estadística, y somos conscientes de que esta ciencia, como sucede con las Matemáticas, no goza de una reputación especialmente buena entre el público general. Recordamos la expresión que hizo popular Mark Twain: ``Hay tres clases de mentiras: mentiras, sucias mentiras y estadísticas''. Desde luego (ver el libro \cite{huff2010lie}), podemos mentir con la Estadística... pero sólo si el que nos escucha no entiende de Estadística.

Nosotros estamos firmemente convencidos de que elevar el nivel de sabiduría estadística de la gente es un deber y una tarea urgente de los sistemas educativos, en los sistemas democráticos.  Una ciudadanía no sólo informada, sino crítica y consciente del valor de la información que recibe, es un ingrediente {\em fundamental} de esos sistemas. Por contra, la ausencia de esos conocimientos no puede sino hacernos más susceptibles al engaño, la manipulación y la demagogia.

Si el conocimiento de la Estadística es importante para cualquier ciudadano, en el caso de quienes se especializan en cualquier disciplina científica o tecnológica, ese conocimiento se vuelve imprescindible. El lenguaje de la Estadística se ha convertido, de hecho, en una parte sustancial del método científico, tal como lo conocemos en la actualidad. Todos los años, con nuestros alumnos, nos gusta hacer el experimento de elegir (al azar, no podía ser de otra manera) unos cuantos artículos de las revistas más prestigiosas en el campo de que se trate, y comprobar que la gran mayoría de ellos emplean el mismo lenguaje estadístico con el que empezamos a familiarizarnos en este curso.

Por todas estas razones nos hemos impuesto la tarea de intentar allanar y hacer simple el acceso a la Estadística. De hecho, vamos a esforzarnos en ser fieles a la máxima de A. Einstein: ``hay que hacer las cosas tan simples como sea posible, pero ni un poco más simples que eso''.

Nuestro interés primordial, en este libro, no es ser rigurosos, y no vamos a serlo. Nos interesa más tratar de llegar al concepto, a la idea que hay detrás del formalismo (a veces muy escondida). Pero, a la vez, no queremos renunciar al mínimo formalismo necesario mostrar algunas de esas ideas, incluso aunque parte de ellas se suelen considerar demasiado ``avanzadas'' para un curso de introducción a la Estadística. Nuestra propia experiencia como aprendices de la Estadística nos ha mostrado, demasiadas veces, que existe una brecha muy profunda entre el nivel elemental y el tratamiento que se hace en los textos que se centran en aspectos concretos de la Estadística aplicada.  Muchos científicos, en su proceso de formación,  pasan de un curso de introducción a la Estadística, directamente al trabajo con las técnicas especializadas que se utilizan en su campo de trabajo. El inconveniente es que, por el camino, se pierde perspectiva. Nos daremos por satisfechos si este libro facilita la transición hacia otros textos de Estadística, más especializados, permitiendo a la vez mantener esa perspectiva más general.


%Se trata, por tanto, para los científicos y profesionales especializados, de una parte básica del lenguaje que habla la {\em tribu}, y si no lo hablamos, no hay esperanza de ser acpetados.


%La Estadística, y en general el análisis de datos, juegan un papel creciente en nuestras vidas. Se dice a menudo que esta es la {\em Era de la Información} y, en los últimos años, la expresión {\em Big Data} se ha convertido en un lugar común de la actualidad científico-tecnológica. Al mismo tiempo, el peso creciente del análisis de datos, no se ve compensado con una mayor difusión de los conocimientos necesarios para entender de qué se trata, y para detectar siquiera las más burdas de esas ``mentiras estadísticas''. La tasa de paro (que en España y en Europa se mide mediante Encuestas de Población Activa). El IPC (Índice de precios al consumo) que elabora el Instituto Nacional de Estadística, y que determina la inflación, las actualizaciones de las pensiones, etc. Los datos de muertes en carretera, que en cada país se calculan con una metodología distinta.
%
%
%Creemos firmemente que la comprensión
%
%Este libro es, entre otras cosas, una parte de nuestro intento por ofrecer una ventana

\section*{Sobre la estructura del libro.}

La Estadística se divide tradicionalmente en varias partes, de modo más o menos arbitrario, porque todas están interconectadas. Basta con revisar el índice de cualquiera de los manuales básicos que aparecen en la Bibliografía. Este libro no es una excepción y esa división resulta evidente en la división del libro en cuatro partes, que describimos brevemente.

\begin{enumerate}
  \item[]{\bf I. Estadística Descriptiva:} esta es la puerta de entrada a la Estadística. En esta parte del libro nuestro objetivo es reflexionar sobre cuál es la información relevante de un conjunto de datos, y aprender a obtenerla en la práctica, cuando disponemos de esos datos.

  \item[]{\bf II. Probabilidad y variables aleatorias:} Pero si la Estadística se limitara a la descripción de los datos que tenemos, su utilidad sería mucho más limitada de lo que realmente es. El verdadero núcleo de la Estadística es la Inferencia, que trata de usar los datos disponibles hacer predicciones (o estimaciones) sobre otros datos que no tenemos. Pero para llegar a la Inferencia, para poder siquiera entender el sentido de esas predicciones, es necesario hablar, al menos de forma básica, el lenguaje de la Probabilidad. En esta parte del curso hemos tratado de incluir el mínimo imprescindible de Probabilidad necesario para que el lector pueda afrontar con éxito el resto de capítulos. Es también la parte del curso que resultará más difícil para los lectores con menor bagaje matemático. La Distribución Binomial y Normal, la relación entre ambas, y el Teorema Central del Límite aparecen en esta parte del curso.

  \item[]{\bf III. Inferencia Estadística:} Como hemos dicho, esta parte del libro contiene lo que a nuestro juicio es el núcleo esencial de ideas que dan sentido y utilidad a la Estadística. Aprovechando los resultados sobre distribuciones muestrales, que son el puente que conecta la Probabilidad con la Inferencia, desarrollaremos las dos ideas básicas de estimación (mediante intervalos de confianza) y contraste de hipótesis. Veremos además, aparecer varias de las distribuciones clásicas más importantes. Trataremos de dar una visión de conjunto de los problemas de estimación y contraste en una amplia variedad de situaciones. Y cerraremos esta parte con el problema de la comparación de un mismo parámetro en dos poblaciones, que sirve de transición natural hacia los métodos que analizan la relación entre dos variables aleatorias. Ese es el contenido de la última parte del curso.

  \item[]{\bf IV. Inferencia sobre la relación entre dos variables:} La parte final del libro contiene una introducción a algunas de las técnicas estadísticas básicas más frecuentemente utilizadas: regresión lineal, anova, contrastes $\chi^2$ y regresión logística. Nos hemos propuesto insistir en la idea de {\em modelo}, porque creemos que puede utilizarse para alcanzar dos objetivos básicos de esta parte de libro. Por un lado, ofrece una visión unificada de lo que, de otra manera, corre el riesgo de parecer un conjunto más o menos inconexo de técnicas (o recetas). La idea de modelo, como siempre al precio de algo de abstracción y formalismo, permite comprender la base común a todos los problemas que aparecen en esta parte del curso. Y, si hemos conseguido ese objetivo, habremos dado un paso firme en la dirección de nuestro segundo objetivo, que consiste en preparar al lector para el salto hacia textos más avanzados de Estadística.

\end{enumerate}

Por supuesto, hay muchos otros temas que hemos dejado fuera. A menudo, muy a nuestro pesar. Para empezar, nos hubiera gustado hablar de Estadística No Paramétrica, del lenguaje Bayesiano, del Diseño de Experimentos o de Análisis Multivariante. Al final del libro, en el Apéndice \ref{apendice:MasAlla}, titulado {\em Más allá de este libro} volveremos sobre este asunto, el de los temas que no hemos cubierto, para dar al menos unas recomendaciones al lector que quiera profundizar en alguno de esos temas. Alguien nos dijo una vez que los libros no se terminan, sino que se abandonan. Somos conscientes de que este libro no está terminado, pero no nos hemos decidido a abandonarlo; todavía no.


\section*{El punto de vista computacional. Tutoriales.}




\section*{¿Cómo usar el libro?}
\label{prefacio:sec:ComoUsar}

Este libro contiene, esencialmente, la parte teórica de un curso de introducción a la Estadística. Pero la Estadística, al menos la Estadística que nosotros queremos hacer, no tiene sentido si no pasamos a la práctica. Y, a su vez, la práctica es inconcebible sin el uso de ordenadores. Por eso la teoría del libro se complementa, de forma inseparable, con una serie de serie de recursos computacionales. En esta sección
{\em Tutoriales}, que a su vez se corresponden con el contenido de las  sesiones prácticas del curso.

\subsection*{Tutoriales.}
\label{prefacio:subsec:Tutoriales}

La primera tarea del lector de este libro, inmediatamente después de leer este Prefacio, debería ser la lectura del Tutorial00. En ese tutorial se explica cómo conseguir e instalar el software necesario para el resto del curso.


En general, cada capítulo del libro se corresponde con un tutorial, y la numeración de capítulos y tutoriales coincide. Sin embargo, los dos primeros Tutoriales, que corresponden a la Parte \ref{parte:EstadisticaDescriptiva} del curso, son especiales. Cada uno de ellos cubre el contenido conjunto de los Capítulos \ref{cap:IntroduccionEstadisticaDescriptiva} y \ref{cap:ValoresCentralesDispersion} de esa parte del curso. Pero en el Tutorial01 se utiliza la hoja de cálculo Calc de OpenOffice, mientras que en el Tutorial02 se usa R. \pendiente{explicar por qué}


\subsection*{¿Por qué R?}

Porque es bueno, bonito y \sout{barato} gratuito. Va en serio. Vale, en lo de bonito tal vez exageramos un poco. Pero a cambio R es {\em free}. En este caso, es una lástima que en español se pierda el doble sentido de la palabra inglesa {\em free}, como libre y gratuito. En inglés podríamos decir (ya es una frase hecha): {\em ``Free as in free speech, free as in free beer''}\footnote{Libre como en Libertad de Expresión, gratis como en cerveza gratis.}

Vamos a tratar de argumentar las razones por las que creemos que R es una gran elección para la enseñanza de la Estadística.

R es un digno heredero de la tradición de la {\em Línea de Comandos}. Los sistemas operativos modernos, tras la aparición de las interfaces gráficas de usuario, se caracterizan por la interacción mediante menús, usando el ratón y apoyándose en una gran cantidad de recursos gráficos. Pero la programación y computación científica siguen dependiendo, de forma esencial, de la interacción en una interfaz basada en texto, en Terminales o Líneas de Comandos.

\subsection*{Página web del libro.}
\label{prefacio:subsec:PaginaWebDelCurso}

Este libro va acompañado de una página web, cuya dirección es
            \begin{center}
                \link{http://www.postdata-statistics.com}{www.postdata-statistics.com}
            \end{center}
Esa página contiene la última versión del libro y los tutoriales, y permite acceder a cuestionarios y otros recursos.

\subsection*{Moodle y Cuestionarios.}
\label{prefacio:subsec:Tutoriales}


\begin{itemize}
    \item Este libro va acompañado de una página web, cuya dirección es
            \begin{center}
                \link{http://www.postdata-statistics.com}{www.postdata-statistics.com}
            \end{center}
        Debes visitarla para obtener la última versión de los tutoriales, acceder a los
    cuestionarios del curso, etc.
    \item El libro está disponible en dos versiones:
        \begin{enumerate}
          \item La versión en color, pensada para visualizarla en una pantalla de ordenador. De hecho hemos tratado de ajustarla para que sea posible utilizar la pantalla de un tablet de 10 pulgadas, pero es el lector quien debe juzgar si ese formato le resulta cómodo.

          \item La versión en blanco y negro, para aquellos usuarios que deseen imprimir alguna parte del libro en una impresora en blanco y negro.  En esta versión las figuras, enlaces, etc. se han adaptado buscando que el resultado sea aceptable en papel.
        \end{enumerate}
        En cualquier caso, y apelando de nuevo al buen juicio del lector, el libro se concibió para usarlo en formato electrónico, puesto que ese es el modo en que resulta más sencillo aprovechar los enlaces y ficheros adjuntos que contiene.

    \item Además, el curso se apoya en Moodle, una plataforma de gestión de cursos on-line. A
        través de la plataforma el lector dispone de material complementario, y cuestionarios de
        corrección automática, para comprobar su comprensión de la teoría.

    \item Nos consta que algunos programas lectores de pdf no muestran los enlaces (en la copia
        en color o los tutoriales, por ejemplo). En particular, desaconsejamos leer esos
        documentos pdf directamente en un navegador de internet. Es mucho mejor guardarlos en el
        disco, y abrirlos con un buen lector. Nuestro favorito, en Windows, es SumatraPDF (bueno,
        bonito y \sout{barato} gratis). En el Tutorial00 encontrarás la dirección de descarga.
        %\link{http://blog.kowalczyk.info/software/sumatrapdf/free-pdf-reader-es.html}{SumatraPDF}

    \item En la misma línea, los documentos pdf de este curso contienen, a veces, enlaces que
        apuntan a otras páginas del documento, y en ocasiones a otros documentos del curso. Por
        ejemplo, el Tutorial03 puede contener una referencia a una página del Tutorial01. Si
        guardas todos los documentos pdf del curso en una misma carpeta, esos enlaces funcionarán
        correctamente, y al usarlos se debería abrir el documento correspondiente, en el punto
        señalado por el enlace. De hecho, te aconsejamos que crees una carpeta en tu ordenador para trabajar con este libro, y que guardes en esa carpeta las versiones en formato pdf de este libro y de todos los tutoriales. Además, y para el trabajo en esos tutoriales, es muy recomendable que crees una subcarpeta llamada {\tt datos}, que nos servirá más adelante para almacenar ficheros auxiliares.

\end{itemize}



%%%%%%%%%%%%%%%%%%%%%%%%%%%%%%%%%%%%%%%%%%%%%%%%%%%%%%%%%%%%%%%%%%%%%%%%%%%%%%%%%%%%%%%%%%%%%%%%%%%%%%
%%%%%%%%%%%%%%%%%%%%%%%%%%%%%%%%%%%%%%%%%%%%%%%%%%%%%%%%%%%%%%%%%%%%%%%%%%%%%%%%%%%%%%%%%%%%%%%%%%%%%%
%%%%%%%%%%%%%%%%%%%%%%%%%%%%%%%%%%%%%%%%%%%%%%%%%%%%%%%%%%%%%%%%%%%%%%%%%%%%%%%%%%%%%%%%%%%%%%%%%%%%%%
%%%%%%%%%%%%%%%%%%%%%%%%%%%%%%%%%%%%%%%%%%%%%%%%%%%%%%%%%%%%%%%%%%%%%%%%%%%%%%%%%%%%%%%%%%%%%%%%%%%%%%
%%%%%%%%%%%%%%%%%%%%%%%%%%%%%%%%%%%%%%%%%%%%%%%%%%%%%%%%%%%%%%%%%%%%%%%%%%%%%%%%%%%%%%%%%%%%%%%%%%%%%%
%%%%%%%%%%%%%%%%%%%%%%%%%%%%%%%%%%%%%%%%%%%%%%%%%%%%%%%%%%%%%%%%%%%%%%%%%%%%%%%%%%%%%%%%%%%%%%%%%%%%%%
%%%%%%%%%%%%%%%%%%%%%%%%%%%%%%%%%%%%%%%%%%%%%%%%%%%%%%%%%%%%%%%%%%%%%%%%%%%%%%%%%%%%%%%%%%%%%%%%%%%%%%
%%%%%%%%%%%%%%%%%%%%%%%%%%%%%%%%%%%%%%%%%%%%%%%%%%%%%%%%%%%%%%%%%%%%%%%%%%%%%%%%%%%%%%%%%%%%%%%%%%%%%%
%%%%%%%%%%%%%%%%%%%%%%%%%%%%%%%%%%%%%%%%%%%%%%%%%%%%%%%%%%%%%%%%%%%%%%%%%%%%%%%%%%%%%%%%%%%%%%%%%%%%%%
%%%%%%%%%%%%%%%%%%%%%%%%%%%%%%%%%%%%%%%%%%%%%%%%%%%%%%%%%%%%%%%%%%%%%%%%%%%%%%%%%%%%%%%%%%%%%%%%%%%%%%
%%%%%%%%%%%%%%%%%%%%%%%%%%%%%%%%%%%%%%%%%%%%%%%%%%%%%%%%%%%%%%%%%%%%%%%%%%%%%%%%%%%%%%%%%%%%%%%%%%%%%%
%%%%%%%%%%%%%%%%%%%%%%%%%%%%%%%%%%%%%%%%%%%%%%%%%%%%%%%%%%%%%%%%%%%%%%%%%%%%%%%%%%%%%%%%%%%%%%%%%%%%%%
%%%%%%%%%%%%%%%%%%%%%%%%%%%%%%%%%%%%%%%%%%%%%%%%%%%%%%%%%%%%%%%%%%%%%%%%%%%%%%%%%%%%%%%%%%%%%%%%%%%%%%
%%%%%%%%%%%%%%%%%%%%%%%%%%%%%%%%%%%%%%%%%%%%%%%%%%%%%%%%%%%%%%%%%%%%%%%%%%%%%%%%%%%%%%%%%%%%%%%%%%%%%%
