% !Mode:: "Tex:UTF-8"

\section*{Introducción a la Probabilidad.}
\label{part02:intro}

\subsection*{Modelos. Fenómenos deterministas y aleatorios.}

Para poner todo lo que viene en perspectiva, nos vamos a detener unas líneas en la idea de {\sf
modelo}\index{modelo}. Básicamente, las ciencias intentan explicar los fen\'omenos que componen la
realidad, que suelen ser muy complicados, debido a la gran cantidad de elementos, muchas veces a
escalas muy distintas a las de nuestra experiencia cotidiana, que interactúan para producir el
resultado que observamos. Por eso resulta necesario hacer simplificaciones y descubrir las reglas
de funcionamiento elementales a partir de las que explicar el resto. Eso es básicamente un modelo:
una simplificación de la realidad, en la que conservamos los rasgos que consideramos esenciales,
para tratar de entender el fenómeno que estamos estudiando. A medida que se va entendiendo un
modelo, se añaden nuevos elementos que lo asemejan m\'as a la realidad. Desde el punto de vista de
la modelizaci\'on, hay dos grandes grupos de fen\'omenos:
\begin{itemize}
    \item Los {\sf fenómenos deterministas} \index{fen\'omeno determinista} son aquellos en los
        que, dadas unas condiciones iniciales, su evoluci\'on futura es totalmente predecible
        (est\'a determinada de antemano). Por ejemplo, cuando lanzamos un proyectil (en un modelo
        en el que despreciamos el rozamiento del aire, el efecto de la rotaci\'on de la 			
        tierra, etc.), una vez conocidas la velocidad con la que se lanza el proyectil y la
        inclinaci\'on respecto de la horizontal, podemos calcular a priori (esto es, predecir)
        con mucha precisión el alcance (a qu\'e distancia caer\'a), la altura m\'axima que
        alcanzar\'a,\dots.
    \item Un {\sf fen\'omeno aleatorio} \index{fen\'omeno aleatorio} 	es aquel que, dadas las
        condiciones iniciales, sabemos el conjunto de posibles resultados, pero no cu\'al de
        ellos suceder\'a. El lanzamiento de una moneda, o de un dado, el sexo de un hijo, son
        algunos ejemplos.
\end{itemize}
Pronto veremos que obtener una muestra de una poblaci\'on (si se hace bien) es un hecho
esencialmente aleatorio. Esto conlleva un cierto grado de incertidumbre (conocemos los posibles
resultados, pero no cu\'al de entre ellos ocurrirá) y la probabilidad es la herramienta adecuada
para lidiar con esa incertidumbre.

\subsection*{El papel de la Probabilidad en la Estadística.}

Hemos venido diciendo desde el principio del curso que el objetivo más importante de la Estadística
es realizar inferencias. Recordemos en que consiste esa idea: estamos interesados en estudiar un
fenómeno que ocurre en una determinada \index{poblaci\'on} {\sf población}. En este contexto,
población no se refiere sólo a seres vivos. Si queremos estudiar la antigüedad del parque móvil de
España, la población la forman todos los vehículos a motor del país (cada vehículo es un
individuo). Si queremos estudiar la dotación tecnológica de los centros de secundaria, la población
la forman todos los institutos, y cada instituto es un individuo de esa población. En general,
resulta imposible, indeseable o inviable estudiar uno por uno todos los individuos de la población.
Por esa razón, lo que hacemos es obtener información sobre una {\sf muestra} de la población. Es
decir, un subconjunto de individuos de la población original, de los que obtenemos información
sobre el fenómeno que nos interesa.

Tenemos que distinguir por lo tanto, en todo lo que hagamos a partir de ahora, qué afirmaciones se
refieren a la población (la colección completa) y cuáles se refieren a la muestra. Los únicos datos
a los que realmente tendremos acceso son los de la muestra (o muestras) que hayamos obtenido. La
muestra nos proporcionará datos sobre alguna variable (o variables) relacionadas con el fenómeno
que estamos estudiando. Es decir, que podemos empezar pensando que en la muestra tenemos, como en
todo lo que hemos hecho hasta ahora un conjunto de $n$ datos,
        \[x_1,x_2,\ldots,x_n.\]

En el ejemplo del parque móvil, podríamos haber obtenido las fichas técnicas de 1000 vehículos (la
población completa consta de cerca de 28 millones de vehículos\footnote{En concreto, 27963880,
según datos de un informe de Anfac del año 2010 (ver enlace  [\,\ref{enlace0000}\,]\label{enlace0000a}) .}). Y una variable que nos puede interesar
para estudiar la antigüedad del parque móvil es el año de matriculación. Así que tendríamos 1000
valores $x_1,\ldots,x_{1000}$, donde cada uno de esos valores representa la antigüedad (en años) de
un vehículo. Con esos 1000 valores podemos calcular una media, que llamaremos la \index{media
muestral} {\sf media muestral}:
    \[\bar x=\dfrac{x_1+x_2+\cdots+x_{1000}}{1000}\]
Naturalmente, si accediéramos a {\em todos} los datos, nos encontraríamos con una lista {\em mucho}
más larga, de alrededor de 28 millones de números:
    \[m_1,m_2,m_3,\ldots,m_{27963880}.\]
Y podríamos hacer la media de todos estos datos, que llamaremos la \index{media poblacional}{\sf
media poblacional}:
    \[\mu=\dfrac{m_1+m_2+m_3+\ldots+m_{27963880}}{27963880}.\]
Los símbolos que hemos elegido no son casuales. Vamos a utilizar siempre $\bar x$ para referirnos a
la media muestral y $\mu$ (la letra griega mu) para referirnos a la media poblacional. Este es un
convenio firmemente asentado entre los usuarios de la Estadística.

Naturalmente, hacer esta media poblacional es mucho más difícil, complicado, caro, etcétera. Y ahí
es donde aparece la idea de inferencia, que se puede formular aproximadamente así, en un sentido
intuitivo:
    \begin{center}
        \fcolorbox{black}{Gris025}{\begin{minipage}{11cm}
            {\sf Si hemos seleccionado esos 1000 coches al \colorbox{lightgrey}{\bf azar} de
            entre los aproximadamente 28 millones posibles, entonces es muy
            \colorbox{lightgrey}{\bf probable} que la media muestral $\bar x$  se parezca mucho a la
            media poblacional $\mu$.}
        \end{minipage}}
    \end{center}
    Hemos destacado en esta frase las palabras azar y probable, porque son la justificación de lo que
    vamos a estar haciendo en los próximos capítulos. Para poder usar la Estadística con rigor
    científico, tenemos que entender qué quiere decir exactamente {\em seleccionar al azar}, y cómo
    se puede {\em medir la probabilidad} de algo. Para esto necesitamos el lenguaje matemático de la
    Teoría de la Probabilidad.

\subsection*{El lenguaje de la Probabilidad.}

La probabilidad nació entre juegos de azar, y sus progenitores incluyen una larga estirpe de
truhanes, fulleros y timadores, junto con algunas de las mentes matemáticas más brillantes de su
época. De esa mezcla de linajes sólo cabía esperar una teoría llena de sorpresas, paradojas,
trampas, cosas que parecen lo que no son... la Probabilidad es muy bonita, y no es demasiado fácil.
De hecho, puede ser muy difícil, y elevarse en grandes abstracciones. Pero le garantizamos al
lector que, como dijimos en la Introducción del libro, vamos a hacer un esfuerzo para hacerle las cosas tan simples como sea posible (y ni un poco más simples).

Una de las razones que, a nuestro juicio, hacen que la Probabilidad resulte más difícil, es que,
sea por razones evolutivas o por cualesquiera  otras razones, el hecho es que los humanos tenemos
una intuición relativamente pobre a la hora de juzgar sobre la probabilidad de distintos
acontecimientos. Por poner un ejemplo, por comparación, nuestra intuición geométrica es bastante
mejor. Pero cuando se trata de evaluar probabilidades, especialmente cuando se trata de sucesos
poco frecuentes, nuestra intuición, en general, nos abandona, y debemos recurrir a las matemáticas
para pisar tierra firme.

A esa dificultad, se suma el hecho de que el nivel matemático del curso va a elevarse en esta
parte, en la que vamos a recurrir, en el Capítulo \ref{cap:Probabilidad} a la Combinatoria, y en el
Capítulo \ref{cap:VariablesAleatorias} al lenguaje de las funciones y del Cálculo. En particular,
necesitaremos la integración. No suponemos que el lector sepa integrar, así que hemos tratado de
incluir, en el Capítulo \ref{cap:VariablesAleatorias}, un tratamiento tan autocontenido del tema
como nos ha sido posible. Afortunadamente, la parte más mecánica (y tediosa) de la tarea de integración se
puede dejar ahora en manos de los ordenadores. Así que, de la misma forma que ya
nadie piensa en aprender a usar una tabla de logaritmos, hemos adoptado la postura de, llegado el
momento, pedir al lector que use el ordenador para calcular tal o cual integral. Esa delegación de los detalles técnicos en las máquinas nos deja libres para concentrarnos en las ideas, que son siempre la parte importante. Aspiramos a que el lector empiece a entender {\em para qué sirve} la integral, aunque no sepa calcular ninguna a mano. Por seguir con la analogía, los logaritmos se calculan con las máquinas, pero eso no nos exime de entender sus propiedades y, sobre todo, cuándo y cómo pueden sernos útiles.

El Capítulo \ref{cap:TeoremaCentralLimite} marca la transición, en la que salimos de la
Probabilidad, para tomar el camino que lleva a la Inferencia, de vuelta a Estadística. Ese
capítulo, y los dos siguientes, son, como hemos dicho, la parte central del curso, donde se
establecen las ideas fundamentales de la Estadística clásica.
