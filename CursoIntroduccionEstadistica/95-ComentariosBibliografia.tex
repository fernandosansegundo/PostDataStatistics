%\section{Libros.}
%
%En esta sección vamos a comentar algunos de los libros que aparecen en la Bibliografía (ver página \pageref{bibliografia}), agrupados de forma más o menos coherente en distintas categorías.
%
%\subsection*{Otros cursos de introducción a la Estadística.}
%
%\begin{itemize}
%  \item OpenIntro \cite{david2010openintro}
%  \item {\em Estadística Básica para estudiantes de ciencias}, de Gorgas, Cardiel y Zamorano. Se puedes descargar en versión electrónica desde \link{http://www.ucm.es/info/Astrof/users/jaz/estadistica.html}{esta página}
%  \item {\em La estadística en Comic}. A fecha de hoy (Junio de 2013) la editorial parece
%	haber cerrado. Incluimos en la bibliografía la versión en inglés.
%
%\end{itemize}
%
%\subsection*{Estadística Matemática.}
%
%\begin{itemize}
%  \item {\em Mathematical Statistics}, de Jun Shao.
%
%  \item Rice. \cite{rice2007mathematical}
%
%  \item De la Horra \cite{horra2003estadistica}
%\end{itemize}
%
%
%\subsection*{Libros de divulgación sobre Análisis de Datos, Estadística y Probabilidad.}
%
%
%\begin{itemize}
%  \item {\em The signal and the noise}, de Nate Silver, y comentar su papel en las elecciones presidenciales de 2010 en Estados Unidos. Decir también algo sobre Bayesianos y frecuentistas.
%
%  \item {\em How to Lie with Statistics}.
%
%  \item Dawkins {\em El Gen Egoísta}.
%
%  \item Monod, sobre el método científico.
%
%\end{itemize}
%
%\subsection*{Bioestadística.}
%
%\begin{itemize}
%  \item Quinn Keough \cite{quinn2002experimental}
%  \item Rosner \cite{quinn2002experimental}
%  \item Milton \cite{milton2001}
%\end{itemize}
%
