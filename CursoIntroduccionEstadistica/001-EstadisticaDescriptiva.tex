% !Mode:: "Tex:UTF-8"
% 001-EstadisticaDescriptiva

\section*{Introducción a la Estadística Descriptiva.}
\label{part01:intro}

Como hemos dicho en la Introducción, la Estadística Descriptiva es la puerta de entrada a la
Estadística. En nuestro trabajo o, aún más en general, en nuestra experiencia diaria, las
personas nos hemos ido convirtiendo, de forma creciente, en recolectores ávidos de datos. Nuestro
hambre de datos se debe a que hemos ido creando cada vez más formas de usarlos, para iluminar
nuestra comprensión de algún fenómeno, y para orientar nuestras decisiones.

Pero antes de llegar a ese punto, y poder usar la información para decidir de forma eficaz, tenemos
que ser capaces de tomar los datos, que son {\em información en bruto} y transformarlos en {\em
información estructurada}. En particular, tenemos que desarrollar técnicas para describir, resumir,
y representar esos datos. Por un lado, para poder aplicarles métodos avanzados de análisis. En este
curso vamos a presentar los más básicos de esos métodos de análisis de datos. Por otro lado,
queremos poder {\em comunicar} a otros la información que contienen esos datos. Por ejemplo,
utilizando técnicas gráficas, de visualización.

Todos esos métodos y técnicas, que nos permiten transformar y describir los datos, forman parte de
la \index{estadística descriptiva} {\sf Estadística Descriptiva}. Así que la Estadística
Descriptiva se encarga del trabajo directo con los {\em datos, a los que tenemos acceso}, y con los
que podemos hacer operaciones. Una parte del proceso incluye operaciones matemáticas, con su
correspondiente dósis de abstracción. Pero, puesto que la Estadística Descriptiva es uno de los
puntos de contacto de la Estadística con el mundo real, también encontraremos muchos problemas
prácticos. Y en particular, en la era de la informatización, muchos problemas de índole
computacional, del tipo ``¿cómo consigo que el ordenador haga eso?''. No queremos, en cualquier caso,
refugiarnos en las matemáticas, obviando esa parte práctica del trabajo. Procesar los datos requiere de
nosotros, a menudo, una cierta soltura con las herramientas computacionales, y el dominio de
algunos trucos del oficio. En la parte más práctica del curso, los Tutoriales, dedicaremos tiempo a
esta tarea.

En esta parte del libro vamos a conocer a algunos actores, protagonistas de la Estadística, que nos
acompañarán a lo largo de todo el curso: la media, la varianza, las frecuencias y percentiles, etc.
Vamos a tocar, siquiera brevemente, el tema de la visualización y representación gráfica de datos.
Hay tanto que decir en ese terreno, que pedimos disculpas al lector por adelantado por lo elemental
de las herramientas que vamos a presentar. Entrar con más profundidad en esta materia exigiría un
espacio del que no disponemos. Como, por  otra parte, nos sucederá más veces a lo largo del curso.
No obstante, sí hemos incluido una breve visita a las nociones de precisión y exactitud, y a la
vertiente más práctica del trabajo con cifras significativas, porque, en nuestra experiencia, a
menudo causa dificultades a los principiantes.

\subsection*{Población y muestra.}

También hemos dicho que todas las partes en que se divide la Estadística están interconectadas
entre sí. Y no sabríamos cerrar esta introducción a la primera parte del libro, especialmente por
ser la primera, sin tratar de tender la vista hacia esas otras partes, que nos esperan más
adelante. Así que vamos a extendernos un poco más aquí, para intentar que el lector tenga un poco
más de perspectiva.

Como hemos dicho, la Estadística Descriptiva trabaja con datos a los que tenemos acceso. Pero, en
muchos casos, esos datos corresponden a una \index{muestra} {\sf muestra}, es decir, a un
subconjunto (más o  menos pequeño), de una \index{población} {\sf población} (más o menos grande),
que nos gustaría estudiar. El problema es que estudiar toda la población puede ser demasiado
difícil o indeseable, o directamente imposible. En ese caso surge la pregunta ¿hasta qué punto los
datos de la muestra son {\em representativos} de la población? Es decir, ¿podemos usar los datos de
la muestra para {\em inferir}, o {\em predecir} las características de la población completa? La
\index{inferencia estadística} {\sf Inferencia Estadística}, que comenzaremos en la tercera parte del
libro, se encarga de dar sentido a estas preguntas, formalizarlas y responderlas. Y es, sin
discusión, el auténtico núcleo, el alma de la Estadística.

En la Inferencia clásica, por tanto, trataremos de usar la información que la Estadística
Descriptiva extrae de los datos de la muestra para poder hacer predicciones precisas sobre las
propiedades de la población. Algunos ejemplos típicos de la clase de predicciones que queremos hacer son las encuestas electorales, el control de calidad empresarial o los ensayos clínicos, que son prototipos de lo que estamos explicando, y que muestran que la Estadística consigue, a menudo, realizar con éxito esa tarea.

¿Por qué funciona la Inferencia? A lo largo del libro tendremos ocasión de profundizar en esta
discusión. Pero podemos adelantar una primera respuesta: funciona porque, en muchos casos,
cualquier muestra {\em bien elegida} (y ya daremos más detalles de lo que significa esto), es
bastante {\em representativa} de la población. Dicho de otra manera, si pensamos en el conjunto de
todas las posibles muestras bien elegidas que podríamos tomar, la inmensa mayoría de ellas serán
coherentes entre sí, y representativas de la población. Un ingrediente clave en este punto, sobre
el que volveremos, es el enorme tamaño del conjunto de posibles muestras. Así que, si tomamos una
{\em al azar}, casi con seguridad habremos tomado una muestra representativa. Y puesto que hemos
mencionado el {\em azar}, parece evidente que la manera de hacer que estas frases imprecisas se
conviertan en afirmaciones científicas, verificables, es utilizar el lenguaje de la {\sf
Probabilidad}. Por esa razón, necesitamos hablar en ese lenguaje para poder hacer Estadística
rigurosa. Y con eso, tenemos trazado el plan de buena parte de este libro y de nuestro curso.
