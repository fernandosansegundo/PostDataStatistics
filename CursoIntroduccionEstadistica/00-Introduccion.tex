% !Mode:: "Tex:UTF-8"

Creemos que es conveniente que antes de adentrarte en el libro leas esta introducción. Pero, en cualquier caso, {\bf antes de pasar a otro capítulo, no dejes de leer} la sección titulada {\em ¿Cómo usar el libro?}, en la página \pageref{prefacio:sec:ComoUsar}.


\section*{Presentación.}

\noindent Este libro nace de las clases que los autores vienen impartiendo, desde hace algunos años, en cursos de tipo {\em ``Introducción a la Estadística''}, dirigidos a estudiantes de los Grados en Biología, Biología Sanitaria y Química de la Universidad de Alcalá. En nuestras clases nos esforzamos en presentar la Estadística dotándola de un ``relato'', de un hilo argumental. En particular, hemos tratado de evitar una de las cosas que menos nos gustan de muchos libros (y clases) de Matemáticas: no queremos contar la solución antes de que el lector sepa cuál es el problema. Nos gustaría pensar que, al menos, nos hemos acercado a ese objetivo, pero serán los lectores quienes puedan juzgarlo. Al fin y al cabo, nos vamos a embarcar con el lector en un viaje hacia la Estadística, y somos conscientes de que esta ciencia, como sucede con las Matemáticas, no goza de una reputación especialmente buena entre el público general. Recordamos la expresión que hizo popular Mark Twain: ``Hay tres clases de mentiras: mentiras, sucias mentiras y estadísticas''. Desde luego (ver el libro \cite{huff2010lie}), es cierto que podemos mentir con la Estadística... pero sólo si el que nos escucha no entiende de Estadística.

Nosotros estamos firmemente convencidos de que elevar el nivel de sabiduría estadística de la gente es un deber y una tarea urgente de los sistemas educativos.  Una ciudadanía no sólo informada, sino crítica y consciente del valor de la información que recibe, es un ingrediente {\em fundamental} de  los sistemas democráticos (y una palanca del cambio en los que no son). Por contra, la ausencia de esos conocimientos no puede sino hacernos más susceptibles al engaño, la manipulación y la demagogia.

Si el conocimiento de la Estadística es importante para cualquier ciudadano, en el caso de quienes se especializan en cualquier disciplina científica o tecnológica, ese conocimiento se vuelve imprescindible. El lenguaje de la Estadística se ha convertido, de hecho, en una parte sustancial del método científico tal como lo conocemos en la actualidad. Todos los años, con nuestros alumnos, nos gusta hacer el experimento de elegir (al azar, no podía ser de otra manera) unos cuantos artículos de las revistas más prestigiosas en el campo de que se trate y comprobar que la gran mayoría de ellos emplean el mismo lenguaje estadístico con el que empezaremos a familiarizarnos en este curso.

Por todas estas razones nos hemos impuesto la tarea de intentar allanar y hacer simple el acceso a la Estadística. De hecho, vamos a esforzarnos en ser fieles a la máxima que se atribuye a A. Einstein: ``hay que hacer las cosas tan simples como sea posible, pero ni un poco más simples que eso''. Nuestro interés primordial, en este libro, no es ser rigurosos en el sentido matemático del término, y no vamos a serlo. Nos interesa más tratar de llegar al concepto, a la idea que dio lugar al formalismo y que, a veces, queda muy oculta en el proceso de generalización y formalización. Pero, a la vez, no queremos renunciar al mínimo formalismo necesario para mostrar algunas de esas ideas, incluso aunque parte de ellas se suelen considerar demasiado ``avanzadas'' para un curso de introducción a la Estadística. Nuestra propia experiencia como aprendices de la Estadística nos ha mostrado, demasiadas veces, que existe una brecha muy profunda entre el nivel elemental y el tratamiento que se hace en los textos que se centran en aspectos concretos de la Estadística aplicada.  Muchos científicos, en su proceso de formación,  pasan de un curso de introducción a la Estadística, directamente al estudio de las técnicas especializadas que se utilizan en su campo de trabajo. El inconveniente es que, por el camino, se pierde perspectiva. Nos daremos por satisfechos si este libro facilita la transición hacia otros textos de Estadística, más especializados, permitiendo a la vez mantener esa perspectiva más general.

\subsection*{Requisitos: a quién va dirigido este libro. }

Como acabamos de explicar, este libro se gestó pensando en alumnos de los primeros cursos universitarios en titulaciones de ciencias. Partimos, por tanto, de la base de que el lector de este libro ha sido expuesto a un nivel de formalismo matemático como el que es común en los últimos cursos de un bachillerato orientado a ese tipo de estudios. En concreto, esperamos que el lector no se asuste al encontrarse con fórmulas, expresiones y manipulaciones algebraicas sencillas y que no se asuste demasiado (un cierto nivel de desazón es razonable) al encontrarse con funciones elementales como los logaritmos y las funciones trigonométricas, con las representaciones gráficas de esas funciones o con ideas como la derivada y la integral. Y en relación con esto queremos dejar claras dos ideas complementarias:
\begin{itemize}
  \item No queremos engañar al lector: la Estadística es una ciencia profundamente matematizada y los autores de este libro somos, por formación, matemáticos. Así que para seguir adelante será necesario hablar algo de ese idioma. Como nuestros alumnos nos han oído decir a menudo, para un científico hay tres lenguajes ineludibles: el inglés, el lenguaje de la programación y el lenguaje de las matemáticas. No hay ciencia moderna que no dependa de un cierto nivel de competencia lingüística en esos tres idiomas. %Algunos textos recientes sobre {\em Análisis de Datos} o {\em Aprendizaje Automático} pueden transmitir la sensación de que se puede avanzar en ese terreno sin matemáticas, dejando todo el trabajo al ordenador. Simplemente, no es cierto. De nuevo, volvemos sobre la idea de que las cosas no pueden hacerse más simples de lo que son. Y las matemáticas son una herramienta {\em irrenunciable} para tratar con cualquier fenómeno moderadamente complejo. Pensar que podemos sustituir esa complejidad por el mero uso de métodos de fuerza bruta apoyados en el ordenador sólo puede conducir a ciencia de peor calidad.
  \item Afortunadamente los ordenadores nos permiten en la actualidad delegar en ellos buena parte del trabajo matemático más tedioso. En particular, las capacidades simbólicas de los ordenadores actuales los convierten en herramientas que van mucho más allá de una calculadora ennoblecida. Si el lector aún no entiende a qué nos referimos, le pedimos un poco de paciencia. Las cosas quedarán mucho más claras al avanzar por los primeros capítulos del libro. Este libro, y el curso al que sirve de guía, se ha diseñado buscando en todo momento que las matemáticas sean una herramienta y no un obstáculo. Un ejemplo: cuando queremos localizar el máximo valor de una función sencilla en un intervalo a menudo recurrimos a dibujar la gráfica de la función con el ordenador y a estimar ese valor máximo simplemente mirando la gráfica. Un enfoque más tradicional y formalista diría que para hacer esto debemos derivar la función, buscar los ceros de la derivada, etc. Desde luego que se puede hacer eso. Pero nuestro enfoque en el trabajo con los alumnos es que en ese caso es bueno seguir usando el ordenador para obtener, de forma simbólica, la ecuación de la derivada y la expresión de sus soluciones. El ordenador acompaña y facilita enormemente nuestro trabajo matemático. En ese sentido creemos que aún está por llegar el auténtico impacto del uso de los ordenadores en la forma en la que nos acercamos a las matemáticas.
\end{itemize}
En el resto de esta introducción el lector encontrará algunos detalles adicionales sobre la forma en que hemos tratado de implementar estas ideas.

%En ese sentido creemos que aún está por llegar el auténtico impacto del uso de los ordenadores en la forma en la que nos acercamos a las matemáticas. Año tras año, al empezar el curso algunos alumnos se nos acercaban tímidamente al final de las primeras clases, para expresar su temor de que las matemáticas que habían aprendido en la Educación Secundaria no les hubieran preparado de forma adecuada para el nivel de este curso. Y podemos decir con satisfacción que año tras año hemos visto disiparse esos temores a las pocas semanas de trabajo, a medida que nuestros alumnos iban descubriendo en el ordenador un aliado para poder concentrarse en las ideas sin perderse en el formalismo.  Nos gustará creer que dentro de unos años esa transición se producirá en los cursos


\section*{Sobre la estructura del libro.}

\noindent La Estadística se divide tradicionalmente en varias partes. Para percibir esa división basta con revisar el índice de cualquiera de los manuales básicos que aparecen en la Bibliografía. Este libro no es una excepción y esa división resulta evidente en la división del libro en cuatro partes. Y aunque esa división resulta siempre más o menos arbitraria, porque todas las partes están interconectadas, conserva una gran utilidad para estructurar lo que al principio hemos llamado el {\em relato} de la Estadística. Veamos por tanto un primer esbozo de la trama:

\begin{enumerate}
  \item[]{\bf I. Estadística Descriptiva:} esta es la puerta de entrada a la Estadística. En esta parte del libro nuestro objetivo es reflexionar sobre cuál es la información relevante de un conjunto de datos, y aprender a obtenerla en la práctica, cuando disponemos de esos datos. Las ideas que aparecen en esta parte son muy sencillas, pero fundamentales en el pleno sentido de la palabra. Todo lo que vamos a discutir en el resto del libro reposa sobre esas pocas ideas básicas.

  \item[]{\bf II. Probabilidad y variables aleatorias:} si la Estadística se limitara a la descripción de los datos que tenemos, su utilidad sería mucho más limitada de lo que realmente es. El verdadero núcleo de la Estadística es la Inferencia, que trata de usar los datos disponibles para hacer predicciones (o estimaciones) sobre otros datos que no tenemos. Pero para llegar a la Inferencia, para poder siquiera entender el sentido de esas predicciones, es necesario hablar, al menos de forma básica, el lenguaje de la Probabilidad. En esta parte del libro hemos tratado de incluir el mínimo imprescindible de Probabilidad necesario para que el lector pueda afrontar con éxito el resto de capítulos. Es también la parte del libro que resultará más difícil para los lectores con menor bagaje matemático. La Distribución Binomial y Normal, la relación entre ambas, y el Teorema Central del Límite aparecen en esta parte del libro.

  \item[]{\bf III. Inferencia Estadística:} como hemos dicho, esta parte del libro contiene lo que a nuestro juicio es el núcleo esencial de ideas que dan sentido y utilidad a la Estadística. Aprovechando los resultados sobre distribuciones muestrales, que son el puente que conecta la Probabilidad con la Inferencia, desarrollaremos las dos ideas básicas de estimación (mediante intervalos de confianza) y contraste de hipótesis. Veremos además, aparecer varias de las distribuciones clásicas más importantes. Trataremos de dar una visión de conjunto de los problemas de estimación y contraste en una amplia variedad de situaciones. Y cerraremos esta parte con el problema de la comparación de un mismo parámetro en dos poblaciones, que sirve de transición natural hacia los métodos que analizan la relación entre dos variables aleatorias. Ese es el contenido de la última parte del libro.

  \item[]{\bf IV. Inferencia sobre la relación entre dos variables:} la parte final del libro contiene una introducción a algunas de las técnicas estadísticas básicas más frecuentemente utilizadas: regresión lineal, Anova, contrastes $\chi^2$ y regresión logística. Nos hemos propuesto insistir en la idea de {\em modelo}, porque creemos que puede utilizarse para alcanzar dos objetivos básicos de esta parte de libro. Por un lado, ofrece una visión unificada de lo que, de otra manera, corre el riesgo de parecer un conjunto más o menos inconexo de técnicas (o recetas). La idea de modelo, como siempre al precio de algo de abstracción y formalismo, permite comprender la base común a todos los problemas que aparecen en esta parte del libro. Y, si hemos conseguido ese objetivo, habremos dado un paso firme en la dirección de nuestro segundo objetivo, que consiste en preparar al lector para el salto hacia textos más avanzados de Estadística. Esta parte del curso trata, por tanto, de ser una rampa de lanzamiento hacia ideas más abstractas pero también más ambiciosas. Para hacer esto hemos optado por limitar nuestro estudio a un tipo especialmente sencillo de modelos: aquellos en los que existe una variable respuesta y, lo que es más importante, una única variable explicativa. A nuestro juicio, el lugar natural para afrontar los problemas multivariable (con varias variables explicativas) es un segundo curso de Estadística, que además puede aprovecharse para cerrar el foco sobre un campo concreto: Biología, Economía, Psicología, etc. Naturalmente, esta decisión deja fuera del alcance de este libro algunos problemas muy interesantes. Pero basándonos en nuestra experiencia docente creemos que los principiantes en el aprendizaje de la Estadística pueden beneficiarse de un primer contacto como el que les proponemos aquí.

\end{enumerate}

Como hemos dicho, hay muchos otros temas que hemos dejado fuera o que sólo hemos comentado muy brevemente. A menudo, muy a nuestro pesar. Para empezar, nos hubiera gustado hablar, por citar algunos temas, de {\em Estadística No Paramétrica}, de {\em Estadística Bayesiana}, del {\em Diseño de Experimentos}, el {\em Análisis Multivariante} o el {\em Aprendizaje Automático}. La principal razón para no incluirlos es, en primer lugar, una cuestión de tiempo: el número de horas disponibles en nuestros cursos universitarios de Estadística obliga a una selección muy rigurosa, a menudo difícil, de los temas que se pueden tratar. Al final del libro, en el Apéndice \ref{apendice:MasAlla}, titulado {\em Más allá de este libro} volveremos sobre algunos de los temas que no hemos cubierto, para dar al menos unas recomendaciones al lector que quiera profundizar en alguno de esos temas. Alguien nos dijo una vez que los libros no se terminan, sino que se abandonan. Somos conscientes de que este libro no está terminado, pero no nos hemos decidido a abandonarlo; todavía no. En el futuro nos gustaría completarlo, añadiendo capítulos sobre algunos de esos temas. Ese es uno de los sentidos en los que nos gusta considerar este libro como un {\em proyecto abierto}. Para discutir otras posibles interpretaciones de ese término debemos pasar al siguiente apartado.

\section*{El punto de vista computacional. Tutoriales.}
\label{prefacio:subsec:Tutoriales}

\noindent Partimos de dos convicciones, que a primera vista pueden parecer difíciles de reconciliar:
\begin{itemize}
  \item En la actualidad, no tiene sentido escribir un curso como este sin atender a los aspectos computacionales de la Estadística. Creemos que la enseñanza de las Matemáticas (y, en particular, de la Estadística) sale siempre beneficiada de su acercamiento a la Computación. A menudo sucede que la mejor forma de entender en profundidad un método o una idea matemática consiste en tratar de experimentar con ella en un ordenador, e incluso implementarla en un lenguaje de programación. Somos además afortunados, porque las herramientas computacionales actuales nos permiten llevar adelante ese plan de forma muy eficaz.
  \item Al tiempo, los detalles finos de esas herramientas computacionales son inevitablemente perecederos. Hay muchos libros y cursos con títulos como ``Estadística con tal o cual programa''. En muchos casos, basta con unos pocos meses para que aparezca una nueva versión del programa o del sistema operativo, o para que alguna pequeña revolución tecnológica haga obsoletos esos libros.
\end{itemize}
Y sin embargo, las ideas básicas de la Estadística no caducan. ¿Cómo podemos hacer compatible nuestro deseo de atender a la computación, sin caer en la trampa de la {\em obsolescencia programada}? Nuestra respuesta consiste en dividir el curso en dos partes:
\begin{itemize}
  \item El libro propiamente dicho, que contiene los aspectos teóricos, las ideas de la Estadística, cuyo plazo de caducidad es mucho mayor que el de las herramientas tecnológicas que las implementan.
  \item Una colección de {\em Tutoriales}, que contienen los aspectos prácticos y computacionales del curso. Hay un tutorial para cada capítulo del curso, y uno adicional que contiene instrucciones detalladas para instalar el software que vamos a usar.
\end{itemize}
En el libro (es decir, en esta parte teórica del curso que estás leyendo) haremos a menudo referencia a esos tutoriales, porque el trabajo práctico debe acompañar en paralelo a nuestro recorrido por las ideas teóricas. Pero nos hemos esmerado en escribir un libro que sea tan {\em neutral} desde el punto de vista del software como nos fuera posible. Eso no significa que nosotros no tengamos predilección por algunas herramientas concretas (enseguida daremos más detalles). Pero nuestro objetivo ha sido dejar la puerta abierta para que otras personas puedan, tomando el libro como base, escribir sus propios tutoriales adaptados a una selección de herramientas computacionales distinta (en todo o parte) de la nuestra.

Dicho esto, las herramientas computacionales que más nos gustan son las que se basan en una interfaz clásica de terminal, o {\em línea de comandos}, basadas en texto y típicas de los lenguajes de programación. Los lenguajes R (ver la referencia \cite{Rcite}) y Python (referencia \cite{Rossum:1995:PRM:869369}) son dos ejemplos claros de ese tipo de herramientas. Las preferimos frente a las herramientas basadas en interfaces gráficas (es decir, menús en los que seleccionamos opciones con el ratón) por varias razones. En primer lugar, y desde el punto de vista pedagógico, porque la experiencia nos ha convencido de que el refuerzo mutuo entre las Matemáticas y la Computación es máximo cuando se usan esas herramientas y el estudiante se esfuerza en {\em programar} las soluciones de los problemas. La resolución de problemas es, como siempre lo ha sido, el ingrediente clave en la enseñanza de las Matemáticas. Y la Programación es una de las mejores encarnaciones posibles de esa idea de resolución de problemas. Además, las interfaces basadas en texto tienen ventajas adicionales desde el punto de vista de la productividad. Y, en un terreno que nos resulta especialmente atractivo, esas herramientas basadas en texto hacen especialmente sencillo acercarse a la idea de {\em Investigación Reproducible} (en inglés, {\em Reproducible Research}, ver el enlace [\,\ref{enlace0000-1}\,]\label{enlace0000a-1}), sobre la que nos extenderemos en alguno de los tutoriales del curso.

Eso no significa, en modo alguno, que minusvaloremos las herramientas gráficas. Muy al contrario. En primer lugar, porque se pueden usar interfaces de línea de comando para producir resultados gráficos de gran calidad. Y en segundo lugar, porque nuestro catálogo de herramientas preferidas incluye desde hace tiempo programas como GeoGebra (ver el enlace [\,\ref{enlace0000-11}\,]\label{enlace0000a-11}), que son otra bendición moderna desde el punto de vista de la enseñanza y visualización matemáticas.

De acuerdo con todo lo anterior, nuestra version inicial de los tutoriales utiliza, como herramienta básica, el lenguaje de programación R, complementado con otras herramientas auxiliares como GeoGebra. ¿Por qué R? Porque es bueno, bonito y \sout{barato} gratuito. Va en serio. Vale, en lo de bonito tal vez exageramos un poco. Pero a cambio R es {\em free}. En este caso, es una lástima que en español se pierda el doble sentido de la palabra inglesa {\em free}, como libre y gratuito. En inglés podríamos decir (ya es una frase hecha): {\em ``Free as in free speech, free as in free beer''}\footnote{Libre como en {\em Libertad de Expresión}, gratis como en {\em cerveza gratis}.} R reúne todas las virtudes que hemos asociado con las herramientas basadas en línea de comandos. Pero, insistimos, la parte teórica del curso trata de ser, como diríamos en inglés, {\em software agnostic}. Nuestra selección de herramientas se basa en programas que, además de las características anteriores, son de código abierto, fácilmente accesibles desde Internet, y multiplataforma (con versiones para los principales sistemas operativos). De esa forma, confiamos en que ningún lector tendrá problemas para acceder a esas herramientas.

Un par de puntualizaciones más sobre nuestra estructura de teoría/tutoriales.
\begin{itemize}
  \item En nuestra práctica docente universitaria, los alumnos acuden por un lado a clases magistrales, que se complementan con sesiones prácticas en aulas con ordenadores. Los tutoriales surgieron como un guión para esas clases prácticas. Pero este libro se ha escrito en pleno auge de la formación online, y somos conscientes de que hay una demanda creciente de materiales adecuados para ese tipo de enseñanza. Al diseñar los tutoriales, lo hemos hecho con la intención de que puedan usarse para el estudio autónomo, pero que también puedan servir de base para unas clases prácticas presenciales de formato más clásico.

  \item Los propios tutoriales incorporan los ejercicios del curso. La parte teórica del curso (lo que llamamos ``el libro'') no incluye ejercicios. En relación con esto, referimos al lector a la sección de esta Introducción sobre la página web del libro, donde encontrará ejercicios adicionales.
\end{itemize}


\section*{¿Cómo usar el libro?}
\label{prefacio:sec:ComoUsar}

\noindent Esta sección describe los aspectos más prácticos del trabajo con el libro.

\subsection*{Tutorial-00: descarga e instalación del software necesario. Guías de trabajo.}

\noindent {\bf La primera tarea} del lector de este libro, tras terminar de leer esta Introducción, debería ser la lectura del Tutorial00. En ese tutorial preliminar se explica cómo conseguir e instalar el software necesario para el resto del curso. Al final del Tutorial00 se explica cuál es el siguiente paso que el lector debe dar, tras instalar el software cómo se describe en ese tutorial.


\subsection*{Página web del libro.}
\label{prefacio:subsec:PaginaWebDelCurso}

\noindent Este libro va acompañado de una página web, cuya dirección es
            \begin{center}
                \link{http://www.postdata-statistics.com}{www.postdata-statistics.com}
            \end{center}
Esa página contiene la última versión disponible del libro, los tutoriales y el resto de los materiales asociados. En particular, permite acceder a una colección de cuestionarios que el lector puede utilizar para comprobar su comprensión de los conceptos y métodos que se presentan en el curso. En cualquier caso, ten en cuenta que si estás usando este libro en la universidad, es posible que tu profesor te de instrucciones adicionales sobre la forma de acceder a los materiales adecuados para ti.

\subsection*{Formatos del Libro. Estructura de directorios para los ficheros del curso.}

\noindent El libro está disponible en dos versiones:
\begin{enumerate}
  \item La versión en color, pensada para visualizarla en una pantalla de ordenador. De hecho, hemos tratado de ajustarla para que sea posible utilizar la pantalla de un tablet de 10 pulgadas, pero es el lector quien debe juzgar si ese formato le resulta cómodo.

  \item La versión en blanco y negro, para aquellos usuarios que deseen imprimir alguna parte del libro en una impresora en blanco y negro.  En esta versión las figuras, enlaces, etc. se han adaptado buscando que el resultado sea aceptable en papel.
\end{enumerate}
En cualquier caso, y apelando de nuevo al buen juicio del lector, el libro se concibió para usarlo en formato electrónico, puesto que ese es el modo en que resulta más sencillo aprovechar los enlaces y ficheros adjuntos que contiene.

Nos consta que algunos programas lectores de pdf no muestran los enlaces (en la copia en color o los tutoriales, por ejemplo). En particular, desaconsejamos leer esos documentos pdf directamente en un navegador de Internet. Es mucho mejor guardarlos en el disco, y abrirlos con un buen lector. En el Tutorial00 encontrarás la dirección de descarga de alguno de esos programas.

En la misma línea, los documentos pdf de este curso contienen, a veces, enlaces que apuntan a otras páginas del documento, y en ocasiones a otros documentos del curso. Por ejemplo, el Tutorial03 puede contener una referencia a una página del Tutorial01. Si guardas todos los documentos pdf del curso en una misma carpeta, esos enlaces funcionarán correctamente, y al usarlos se debería abrir el documento correspondiente, en el punto señalado por el enlace. De hecho, te aconsejamos que crees una carpeta en tu ordenador para trabajar con este libro, y que guardes en esa carpeta las versiones en formato pdf de este libro y de todos los tutoriales. Además, y para facilitar el trabajo en esos tutoriales, es muy recomendable que crees una subcarpeta llamada {\tt datos}, que nos servirá más adelante para almacenar ficheros auxiliares.

